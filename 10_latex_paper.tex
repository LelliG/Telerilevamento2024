\documentclass[12 pt]{article} %comment. %[]per la struttura, uso 12 pt
\usepackage{graphicx} % Required for inserting images
\usepackage{hyperref}
\title{LaTeX doc}
\author{Gianluca Lelli}
%\date{} se commento, prende data dal pc in automatico

\begin{document} %inizia il documento.Sempre link con \end{}

\maketitle %raccoglie informazione per il titolo e impagina in prima pagina.
\tableofcontents %sommario
\newpage
\section{Introduction}%con * non si numera.\section*{Introduction}
\label{sec:intro}
There's a lady who's sure all that glitters is gold
And she's buying a stairway to Heaven
When she gets there she knows, if the stores are all closed
With a word she can get what she came for
Ooh, ooh, and she's buying a stairway to Heaven
There's a sign on the wall, but she wants to be sure
'Cause you know sometimes words have two meanings
In a tree by the brook, there's a songbird who sings
Sometimes all of our thoughts are misgiven
Ooh, it makes me wonder
Ooh, makes me wonder
%riga vuota crea un nuovo paragrafo identato.

There's a lady who's sure all that glitters is gold
And she's buying a stairway to Heaven
When she gets there she knows, if the stores are all closed
With a word she can get what she came for
Ooh, ooh, and she's buying a stairway to Heaven
There's a sign on the wall, but she wants to be sure
'Cause you know sometimes words have two meanings
In a tree by the brook, there's a songbird who sings
Sometimes all of our thoughts are misgiven
Ooh, it makes me wonder
Ooh, makes me wonder

\noindent There's a lady who's sure all that glitters is gold
And she's buying a stairway to Heaven
When she gets there she knows, if the stores are all closed
With a word she can get what she came for
Ooh, ooh, and she's buying a stairway to Heaven
There's a sign on the wall, but she wants to be sure
'Cause you know sometimes words have two meanings
In a tree by the brook, there's a songbird who sings
Sometimes all of our thoughts are misgiven
Ooh, it makes me wonder
Ooh, makes me wonder %se volessi mai non indentare, tipo per funzioni matematiche.



\section{Methods}
\subsection{Study Area}
Ooh, it makes me wonder
Ooh, makes me wonder
\subsection{Algorithms}
The Equation \ref{eq:somma} uses was the following:
\begin{equation}
    T = \sum p_1\label{eq:somma}
\end{equation}
In this thesis we made use of Equation \ref{eq:newton}:
\begin{equation}
    F_1=F_2= \sqrt[n]{G\frac{m_1 \times m_2}{d^2}}%\frac per frazioni
    \label{eq:newton} %insersco eq giusto per ricordare di che oggetto si tratta 
\end{equation}

\section{Results}
\section{Discussion}
Our results are in line with previous paper, introduce in section \ref{sec:intro}.
\end{document}

